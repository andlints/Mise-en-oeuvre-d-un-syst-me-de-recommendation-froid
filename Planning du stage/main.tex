\documentclass{article}
\usepackage[utf8]{inputenc}
\usepackage{geometry}
\geometry{hmargin=1cm, vmargin=1cm}

\title{Planning du stage}
\author{ANDRIAMAHERISOA LIANTSOA }
\date{June 2020}

\begin{document}

\maketitle

\section{Première semaine}

    \textbf{Objectifs : }\newline\\
    La première semaine de stage a été la découverte du sujet ainsi que plusieurs sources : articles, videos, ... qui seront très importants par la suite dans le déroulement des travaux.\newline
    C'était également l'occasion d'apprehender les outils indispensables aux implementations de code mais a ceuxi qui vont servire dans la rédaction des rapports qui vont suivre.\newline
    On peut dire que c'est plus une spécification des besoins\\\\
    \textbf{Details : }\newline\\
    Jour 1 : \\\\
        - Prise en main du sujet\\
        - installation de la librairie POT\\
        - lecture de l'article1: "Wasserstein Collaborative Filtering for Item Cold-start Recommendation"\\
        - création de depot github\\ 
        - premier readme sur le résumer de l'article\\\\
    Jour 2: \\\\
        - suite de la lecture de l'article1\\
        - recherche de documetation sur le "Transport Optimal" (TO)\\
        - videos "Marco Cuturi"
        - prise de notes sur le TO\\\\
    Jour 3: \\\\
        - videos "Macro Cututi"\\
        - prise de note sur le TO\\
        - recherche d'autres articles dont: "Computational Optimal Transport", "Le Transport Optimal, couteau suisse pour la Data Science"\\
        - compréhension des articles et prise de notes\\\\
    Jour 4: \\\\
        - prise en main et compréhension du fonctionnement de latex\\
        - commencement de la rédaction du résumer sur le TO\\\\
    Jour 5: \\\\
        - suite de la rédaction du résumer sur le TO\\\\
    \textbf{Difficultés : }\newline\\
    - le manque de tuto sur le TO\\
    - la prise en main de Latex\\
    -.\\\\
    
\section{Deuxième semaine}

    \textbf{Objectifs : }\newline\\
    Continuer à lier et à comprendre le sujet, le fonctionnement des méthodes et leur manipulation. Le principe de TO appliqué à notre problème de système de recommandation à froid.
    Appréhender le lien qui peut exister entre les deux.
    Rédaction de rapport sur ce sujet de manière plus théorique illustrant les détails du procéssus.\\
    Il y aura: lecture d'articles, prise de notes, Videos, rédaction de rapport,....\\
    Définition de problème jouet avec des données empiriques et y faire un rapport experimental.\newline\\
    Jour 1 : \\\\
        - visionnage video: transport optimal et application [Gariel Peyré]\\
        - suite artice (sujet)\\
        - téléchargement et apprehension du dataset "MovieLens" : Visualization.ipynb\\\\
    Jour 2 : \\\\
        - vision de données\\
        - création de certains exemples\\\\
    Jour 3 : \\\\
        - article (sujet)\\
        - essaie d'adaptation datasets et codes\\\\
    Jour 4 : \\\\
        - tentative de création de matrice de coût avec les genres de films\\\\
    Jour 5 : \\\\
        - relecture de l'article\\
        - recherche de méthodes de manipulations des tags genomes\\\\

\section{Troisième semaine}

    \textbf{Objectifs : }\newline\\
    Continuer à lier et à comprendre, le fonctionnement des méthodes et leur manipulation. \\
    Continuer le problème jouet\\
    Procéssus un peu difficile à comprendre\\
    .\newline\\
    
        
\end{document}
